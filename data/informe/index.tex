% Options for packages loaded elsewhere
% Options for packages loaded elsewhere
\PassOptionsToPackage{unicode}{hyperref}
\PassOptionsToPackage{hyphens}{url}
\PassOptionsToPackage{dvipsnames,svgnames,x11names}{xcolor}
%
\documentclass[
  letterpaper,
  DIV=11,
  numbers=noendperiod]{scrartcl}
\usepackage{xcolor}
\usepackage{amsmath,amssymb}
\setcounter{secnumdepth}{-\maxdimen} % remove section numbering
\usepackage{iftex}
\ifPDFTeX
  \usepackage[T1]{fontenc}
  \usepackage[utf8]{inputenc}
  \usepackage{textcomp} % provide euro and other symbols
\else % if luatex or xetex
  \usepackage{unicode-math} % this also loads fontspec
  \defaultfontfeatures{Scale=MatchLowercase}
  \defaultfontfeatures[\rmfamily]{Ligatures=TeX,Scale=1}
\fi
\usepackage{lmodern}
\ifPDFTeX\else
  % xetex/luatex font selection
\fi
% Use upquote if available, for straight quotes in verbatim environments
\IfFileExists{upquote.sty}{\usepackage{upquote}}{}
\IfFileExists{microtype.sty}{% use microtype if available
  \usepackage[]{microtype}
  \UseMicrotypeSet[protrusion]{basicmath} % disable protrusion for tt fonts
}{}
\makeatletter
\@ifundefined{KOMAClassName}{% if non-KOMA class
  \IfFileExists{parskip.sty}{%
    \usepackage{parskip}
  }{% else
    \setlength{\parindent}{0pt}
    \setlength{\parskip}{6pt plus 2pt minus 1pt}}
}{% if KOMA class
  \KOMAoptions{parskip=half}}
\makeatother
% Make \paragraph and \subparagraph free-standing
\makeatletter
\ifx\paragraph\undefined\else
  \let\oldparagraph\paragraph
  \renewcommand{\paragraph}{
    \@ifstar
      \xxxParagraphStar
      \xxxParagraphNoStar
  }
  \newcommand{\xxxParagraphStar}[1]{\oldparagraph*{#1}\mbox{}}
  \newcommand{\xxxParagraphNoStar}[1]{\oldparagraph{#1}\mbox{}}
\fi
\ifx\subparagraph\undefined\else
  \let\oldsubparagraph\subparagraph
  \renewcommand{\subparagraph}{
    \@ifstar
      \xxxSubParagraphStar
      \xxxSubParagraphNoStar
  }
  \newcommand{\xxxSubParagraphStar}[1]{\oldsubparagraph*{#1}\mbox{}}
  \newcommand{\xxxSubParagraphNoStar}[1]{\oldsubparagraph{#1}\mbox{}}
\fi
\makeatother

\usepackage{color}
\usepackage{fancyvrb}
\newcommand{\VerbBar}{|}
\newcommand{\VERB}{\Verb[commandchars=\\\{\}]}
\DefineVerbatimEnvironment{Highlighting}{Verbatim}{commandchars=\\\{\}}
% Add ',fontsize=\small' for more characters per line
\usepackage{framed}
\definecolor{shadecolor}{RGB}{241,243,245}
\newenvironment{Shaded}{\begin{snugshade}}{\end{snugshade}}
\newcommand{\AlertTok}[1]{\textcolor[rgb]{0.68,0.00,0.00}{#1}}
\newcommand{\AnnotationTok}[1]{\textcolor[rgb]{0.37,0.37,0.37}{#1}}
\newcommand{\AttributeTok}[1]{\textcolor[rgb]{0.40,0.45,0.13}{#1}}
\newcommand{\BaseNTok}[1]{\textcolor[rgb]{0.68,0.00,0.00}{#1}}
\newcommand{\BuiltInTok}[1]{\textcolor[rgb]{0.00,0.23,0.31}{#1}}
\newcommand{\CharTok}[1]{\textcolor[rgb]{0.13,0.47,0.30}{#1}}
\newcommand{\CommentTok}[1]{\textcolor[rgb]{0.37,0.37,0.37}{#1}}
\newcommand{\CommentVarTok}[1]{\textcolor[rgb]{0.37,0.37,0.37}{\textit{#1}}}
\newcommand{\ConstantTok}[1]{\textcolor[rgb]{0.56,0.35,0.01}{#1}}
\newcommand{\ControlFlowTok}[1]{\textcolor[rgb]{0.00,0.23,0.31}{\textbf{#1}}}
\newcommand{\DataTypeTok}[1]{\textcolor[rgb]{0.68,0.00,0.00}{#1}}
\newcommand{\DecValTok}[1]{\textcolor[rgb]{0.68,0.00,0.00}{#1}}
\newcommand{\DocumentationTok}[1]{\textcolor[rgb]{0.37,0.37,0.37}{\textit{#1}}}
\newcommand{\ErrorTok}[1]{\textcolor[rgb]{0.68,0.00,0.00}{#1}}
\newcommand{\ExtensionTok}[1]{\textcolor[rgb]{0.00,0.23,0.31}{#1}}
\newcommand{\FloatTok}[1]{\textcolor[rgb]{0.68,0.00,0.00}{#1}}
\newcommand{\FunctionTok}[1]{\textcolor[rgb]{0.28,0.35,0.67}{#1}}
\newcommand{\ImportTok}[1]{\textcolor[rgb]{0.00,0.46,0.62}{#1}}
\newcommand{\InformationTok}[1]{\textcolor[rgb]{0.37,0.37,0.37}{#1}}
\newcommand{\KeywordTok}[1]{\textcolor[rgb]{0.00,0.23,0.31}{\textbf{#1}}}
\newcommand{\NormalTok}[1]{\textcolor[rgb]{0.00,0.23,0.31}{#1}}
\newcommand{\OperatorTok}[1]{\textcolor[rgb]{0.37,0.37,0.37}{#1}}
\newcommand{\OtherTok}[1]{\textcolor[rgb]{0.00,0.23,0.31}{#1}}
\newcommand{\PreprocessorTok}[1]{\textcolor[rgb]{0.68,0.00,0.00}{#1}}
\newcommand{\RegionMarkerTok}[1]{\textcolor[rgb]{0.00,0.23,0.31}{#1}}
\newcommand{\SpecialCharTok}[1]{\textcolor[rgb]{0.37,0.37,0.37}{#1}}
\newcommand{\SpecialStringTok}[1]{\textcolor[rgb]{0.13,0.47,0.30}{#1}}
\newcommand{\StringTok}[1]{\textcolor[rgb]{0.13,0.47,0.30}{#1}}
\newcommand{\VariableTok}[1]{\textcolor[rgb]{0.07,0.07,0.07}{#1}}
\newcommand{\VerbatimStringTok}[1]{\textcolor[rgb]{0.13,0.47,0.30}{#1}}
\newcommand{\WarningTok}[1]{\textcolor[rgb]{0.37,0.37,0.37}{\textit{#1}}}

\usepackage{longtable,booktabs,array}
\usepackage{calc} % for calculating minipage widths
% Correct order of tables after \paragraph or \subparagraph
\usepackage{etoolbox}
\makeatletter
\patchcmd\longtable{\par}{\if@noskipsec\mbox{}\fi\par}{}{}
\makeatother
% Allow footnotes in longtable head/foot
\IfFileExists{footnotehyper.sty}{\usepackage{footnotehyper}}{\usepackage{footnote}}
\makesavenoteenv{longtable}
\usepackage{graphicx}
\makeatletter
\newsavebox\pandoc@box
\newcommand*\pandocbounded[1]{% scales image to fit in text height/width
  \sbox\pandoc@box{#1}%
  \Gscale@div\@tempa{\textheight}{\dimexpr\ht\pandoc@box+\dp\pandoc@box\relax}%
  \Gscale@div\@tempb{\linewidth}{\wd\pandoc@box}%
  \ifdim\@tempb\p@<\@tempa\p@\let\@tempa\@tempb\fi% select the smaller of both
  \ifdim\@tempa\p@<\p@\scalebox{\@tempa}{\usebox\pandoc@box}%
  \else\usebox{\pandoc@box}%
  \fi%
}
% Set default figure placement to htbp
\def\fps@figure{htbp}
\makeatother





\setlength{\emergencystretch}{3em} % prevent overfull lines

\providecommand{\tightlist}{%
  \setlength{\itemsep}{0pt}\setlength{\parskip}{0pt}}



 


\KOMAoption{captions}{tableheading}
\makeatletter
\@ifpackageloaded{caption}{}{\usepackage{caption}}
\AtBeginDocument{%
\ifdefined\contentsname
  \renewcommand*\contentsname{Table of contents}
\else
  \newcommand\contentsname{Table of contents}
\fi
\ifdefined\listfigurename
  \renewcommand*\listfigurename{List of Figures}
\else
  \newcommand\listfigurename{List of Figures}
\fi
\ifdefined\listtablename
  \renewcommand*\listtablename{List of Tables}
\else
  \newcommand\listtablename{List of Tables}
\fi
\ifdefined\figurename
  \renewcommand*\figurename{Figure}
\else
  \newcommand\figurename{Figure}
\fi
\ifdefined\tablename
  \renewcommand*\tablename{Table}
\else
  \newcommand\tablename{Table}
\fi
}
\@ifpackageloaded{float}{}{\usepackage{float}}
\floatstyle{ruled}
\@ifundefined{c@chapter}{\newfloat{codelisting}{h}{lop}}{\newfloat{codelisting}{h}{lop}[chapter]}
\floatname{codelisting}{Listing}
\newcommand*\listoflistings{\listof{codelisting}{List of Listings}}
\makeatother
\makeatletter
\makeatother
\makeatletter
\@ifpackageloaded{caption}{}{\usepackage{caption}}
\@ifpackageloaded{subcaption}{}{\usepackage{subcaption}}
\makeatother
\usepackage{bookmark}
\IfFileExists{xurl.sty}{\usepackage{xurl}}{} % add URL line breaks if available
\urlstyle{same}
\hypersetup{
  pdftitle={Comparación R, Python y Julia en ANOVA y Regresión Lineal},
  colorlinks=true,
  linkcolor={blue},
  filecolor={Maroon},
  citecolor={Blue},
  urlcolor={Blue},
  pdfcreator={LaTeX via pandoc}}


\title{Comparación R, Python y Julia en ANOVA y Regresión Lineal}
\author{Antonio Rafael Arias Romero}
\date{}
\begin{document}
\maketitle


\subsection{Introducción}\label{introducciuxf3n}

En el análisis de datos, se nos presentan varias herramientas
fundamentales, los lenguajes de programación son parte de ellas,
fundamentales para investigadores y profesionales de múltiples
disciplinas. Entre ellos R, Python y Julia destacan por su amplia
adopción, riqueza en librerías y capacidad para realizar tareas
sencillas de manipulación de datos hasta complejos análisis estadísticos
y computacionales de alto rendimiento. Sin embargo, cada uno de estos
lenguajes presentan diferencias en sintaxis, paradigmas de ejecución,
optimización interna y ecosistema de paquetes. Estas diferencias afectan
directamente al tiempo de ejecución, al consumo de memoria, a la
precisión numérica y a la facilidad de uso al implementar procedimientos
estadísticos clásicos, como la regresión lineal y el análisis de
varianza (ANOVA). Ante esta diversidad, surge la necesidad de contar con
evidencia empírica y comparativa que guíe al investigador en la elección
de la herramienta más adecuada según sus prioridades (velocidad,
legibilidad del código, reproducibilidad, etc).

\subsection{Objetivos}\label{objetivos}

\begin{itemize}
\tightlist
\item
  Hallar la variación del tiempo de ejecución y el consumo de memoria
  entre R, Python y Julia al ajustar un modelo de regresión lineal
  múltiple y al realizar un análisis de varianza (ANOVA) sobre el mismo
  conjunto de datos.
\item
  La diferencia existente en la precisión numérica de los resultaqdos
  -coeficientes, errores estándaar y valores p- obtenidos por cada
  lenguaje, tomando R como referencia.
\item
  La cantidad de líneas de código y qué grado de complejidad sintáctica
  requiere cada lenguaje para implementar estos procedimientos
  estadísticos clásicos
\end{itemize}

\subsection{Conjunto de datos}\label{conjunto-de-datos}

Todos los experimentos midieron el \textbf{tiempo de ejecución} (p.\,e.,
usando \texttt{system.time} en R o \texttt{time} en Python), el
\textbf{consumo máximo de memoria} (en MB, usando perfiles de memoria
del SO), la \textbf{precisión numérica} (comparando coeficientes,
errores estándar y p-valores frente a R) y la \textbf{facilidad de uso}
(líneas de código y claridad sintáctica). Para ANOVA se empleó el mismo
modelo de regresión con un factor, y para regresión lineal se ajustó un
modelo lineal. Así mismo se hizo uso de un csv el cual todos los
archivos ejecutados leyeron llamada \texttt{data.csv}

\subsubsection{\texorpdfstring{\textbf{Ejecución en
R}}{Ejecución en R}}\label{ejecuciuxf3n-en-r}

En \textbf{R}, la regresión se realiza con la función \texttt{lm()} y el
análisis de varianza con \texttt{aov()} o \texttt{anova()}. Por ejemplo,
\texttt{lm(Y\ \textasciitilde{}\ X1\ +\ X2,\ data\ =\ datos)} ajusta un
modelo lineal múltiple y \texttt{anova(modelo)} devuelve la tabla ANOVA
correspondiente. El código es muy conciso: normalmente se necesita una
línea para ajustar el modelo y otra para inspeccionar los resultados. R
ofrece todos los estadísticos de interés ---coeficientes, errores
estándar, valores p--- mediante \texttt{summary(modelo)}.

\begin{Shaded}
\begin{Highlighting}[]
\CommentTok{\# Leer datos}
\NormalTok{datos }\OtherTok{\textless{}{-}} \FunctionTok{read.csv}\NormalTok{(}\StringTok{"data.csv"}\NormalTok{)}

\CommentTok{\# ANOVA}
\NormalTok{modelo }\OtherTok{\textless{}{-}} \FunctionTok{aov}\NormalTok{(Y }\SpecialCharTok{\textasciitilde{}}\NormalTok{ GRUPO, }\AttributeTok{data =}\NormalTok{ datos)}
\FunctionTok{print}\NormalTok{(}\FunctionTok{summary}\NormalTok{(modelo))}

\CommentTok{\# Guardar tabla ANOVA}
\FunctionTok{write.csv}\NormalTok{(}\FunctionTok{summary}\NormalTok{(modelo)[[}\DecValTok{1}\NormalTok{]], }\StringTok{"resultados\_anova\_R.csv"}\NormalTok{)}

\CommentTok{\# Regresión lineal múltiple}
\NormalTok{modelo }\OtherTok{\textless{}{-}} \FunctionTok{lm}\NormalTok{(Y }\SpecialCharTok{\textasciitilde{}}\NormalTok{ X1 }\SpecialCharTok{+}\NormalTok{ X2, }\AttributeTok{data =}\NormalTok{ datos)}
\FunctionTok{print}\NormalTok{(}\FunctionTok{summary}\NormalTok{(modelo))}

\CommentTok{\# Guardar coeficientes}
\FunctionTok{write.csv}\NormalTok{(}\FunctionTok{summary}\NormalTok{(modelo)}\SpecialCharTok{$}\NormalTok{coefficients, }\StringTok{"resultados\_regresion\_R.csv"}\NormalTok{)}
\end{Highlighting}
\end{Shaded}

Esta salida muestra los coeficientes estimados (β), errores estándar,
valores t y p, e intervalos de confianza. En ANOVA,
\texttt{anova(modelo)} detalla las sumas de cuadrados, grados de
libertad y estadísticos F. En nuestras pruebas, R sirvió como
\textbf{referencia de precisión} para comparar los resultados de Python
y Julia. Además, su sintaxis vectorizada y las funciones estadísticas
integradas permiten resolver ambos análisis con \textbf{pocas líneas de
código} (2--3 por modelo).

\begin{center}\rule{0.5\linewidth}{0.5pt}\end{center}

\subsubsection{\texorpdfstring{\textbf{Ejecución en
Python}}{Ejecución en Python}}\label{ejecuciuxf3n-en-python}

En \textbf{Python}, se usan principalmente \texttt{pandas} para
manipular datos y \texttt{statsmodels} para ajustar modelos
estadísticos. Primero se importan los módulos necesarios:

\begin{Shaded}
\begin{Highlighting}[]
\ImportTok{import}\NormalTok{ pandas }\ImportTok{as}\NormalTok{ pd}
\ImportTok{import}\NormalTok{ statsmodels.formula.api }\ImportTok{as}\NormalTok{ smf}
\ImportTok{import}\NormalTok{ statsmodels.stats.api }\ImportTok{as}\NormalTok{ sms}
\end{Highlighting}
\end{Shaded}

Para ajustar un modelo de regresión, se emplea la interfaz de fórmulas:

\begin{Shaded}
\begin{Highlighting}[]
\CommentTok{\# Leer datos}
\NormalTok{df }\OperatorTok{=}\NormalTok{ pd.read\_csv(}\StringTok{"data.csv"}\NormalTok{)}

\CommentTok{\# Regresión lineal múltiple}
\NormalTok{modelo }\OperatorTok{=}\NormalTok{ smf.ols(}\StringTok{\textquotesingle{}Y \textasciitilde{} X1 + X2\textquotesingle{}}\NormalTok{, data}\OperatorTok{=}\NormalTok{df).fit()}
\BuiltInTok{print}\NormalTok{(modelo.summary())}

\CommentTok{\# Guardar resultados}
\NormalTok{res\_df }\OperatorTok{=}\NormalTok{ pd.DataFrame(\{}
    \StringTok{\textquotesingle{}coef\textquotesingle{}}\NormalTok{: modelo.params,}
    \StringTok{\textquotesingle{}std\_err\textquotesingle{}}\NormalTok{: modelo.bse,}
    \StringTok{\textquotesingle{}pvalue\textquotesingle{}}\NormalTok{: modelo.pvalues}
\NormalTok{\})}
\NormalTok{res\_df.to\_csv(}\StringTok{"resultados\_regresion\_Python.csv"}\NormalTok{, index}\OperatorTok{=}\VariableTok{True}\NormalTok{)}
\end{Highlighting}
\end{Shaded}

Para realizar ANOVA se utiliza \texttt{anova\_lm()} de
\texttt{statsmodels}:

\begin{Shaded}
\begin{Highlighting}[]
\CommentTok{\# ANOVA de un factor}
\NormalTok{modelo }\OperatorTok{=}\NormalTok{ smf.ols(}\StringTok{\textquotesingle{}Y \textasciitilde{} C(GRUPO)\textquotesingle{}}\NormalTok{, data}\OperatorTok{=}\NormalTok{df).fit()}
\NormalTok{anova\_table }\OperatorTok{=}\NormalTok{ sms.anova\_lm(modelo, typ}\OperatorTok{=}\DecValTok{2}\NormalTok{)}
\BuiltInTok{print}\NormalTok{(anova\_table)}

\CommentTok{\# Guardar tabla ANOVA}
\NormalTok{anova\_table.to\_csv(}\StringTok{"resultados\_anova\_Python.csv"}\NormalTok{)}
\end{Highlighting}
\end{Shaded}

La combinación \texttt{pandas} + \texttt{statsmodels} es muy flexible y
produce resultados comparables a R: \texttt{params} para coeficientes,
\texttt{bse} para errores estándar y \texttt{pvalues} para contrastes de
hipótesis. Aunque Python requiere unas líneas adicionales para importar
módulos y definir la fórmula, la sintaxis sigue siendo clara y
declarativa. En los benchmarks, Python mostró un \textbf{uso de memoria
más alto} (por el motor de \texttt{numpy} y \texttt{pandas}) y tiempos
de ejecución similares o ligeramente mayores que R.

\begin{center}\rule{0.5\linewidth}{0.5pt}\end{center}

\subsubsection{\texorpdfstring{\textbf{Ejecución en
Julia}}{Ejecución en Julia}}\label{ejecuciuxf3n-en-julia}

\textbf{Julia} combina alto rendimiento con una sintaxis expresiva,
similar a R. Para realizar regresión o ANOVA se suele usar el paquete
\texttt{GLM.jl} junto a \texttt{DataFrames}:

\begin{Shaded}
\begin{Highlighting}[]
\ImportTok{using} \BuiltInTok{CSV}\NormalTok{, }\BuiltInTok{DataFrames}\NormalTok{, }\BuiltInTok{GLM}\NormalTok{, }\BuiltInTok{StatsModels}

\CommentTok{\# Leer datos}
\NormalTok{df }\OperatorTok{=}\NormalTok{ CSV.}\FunctionTok{read}\NormalTok{(}\StringTok{"data.csv"}\NormalTok{, DataFrame)}

\CommentTok{\# Regresión lineal múltiple}
\NormalTok{modelo }\OperatorTok{=} \FunctionTok{lm}\NormalTok{(}\PreprocessorTok{@formula}\NormalTok{(Y }\OperatorTok{\textasciitilde{}}\NormalTok{ X1 }\OperatorTok{+}\NormalTok{ X2), df)}
\FunctionTok{println}\NormalTok{(}\FunctionTok{coeftable}\NormalTok{(modelo))}

\CommentTok{\# Guardar coeficientes}
\NormalTok{CSV.}\FunctionTok{write}\NormalTok{(}\StringTok{"resultados\_regresion\_Julia.csv"}\NormalTok{, }\FunctionTok{coeftable}\NormalTok{(modelo))}
\end{Highlighting}
\end{Shaded}

En el caso del ANOVA unifactorial, se puede calcular la tabla ANOVA de
forma manual, dado que \texttt{GLM.jl} por sí solo no genera
automáticamente tablas de sumas de cuadrados tipo II/III:

\begin{Shaded}
\begin{Highlighting}[]
\ImportTok{using} \BuiltInTok{CSV}\NormalTok{, }\BuiltInTok{DataFrames}\NormalTok{, }\BuiltInTok{GLM}\NormalTok{, }\BuiltInTok{StatsModels}

\CommentTok{\# Leer datos}
\NormalTok{df }\OperatorTok{=}\NormalTok{ CSV.}\FunctionTok{read}\NormalTok{(}\StringTok{"data.csv"}\NormalTok{, DataFrame)}

\CommentTok{\# Ajustar modelo}
\NormalTok{model }\OperatorTok{=} \FunctionTok{lm}\NormalTok{(}\PreprocessorTok{@formula}\NormalTok{(Y }\OperatorTok{\textasciitilde{}}\NormalTok{ GRUPO), df)}

\CommentTok{\# Calcular ANOVA manual}
\NormalTok{ss\_residual }\OperatorTok{=} \FunctionTok{deviance}\NormalTok{(model)}
\NormalTok{df\_residual }\OperatorTok{=} \FunctionTok{dof\_residual}\NormalTok{(model)}
\NormalTok{terms }\OperatorTok{=} \FunctionTok{coeftable}\NormalTok{(model)}

\NormalTok{df\_anova }\OperatorTok{=} \FunctionTok{DataFrame}\NormalTok{(}
\NormalTok{    Source }\OperatorTok{=}\NormalTok{ [}\StringTok{"GRUPO"}\NormalTok{, }\StringTok{"Residual"}\NormalTok{],}
\NormalTok{    DOF }\OperatorTok{=}\NormalTok{ [}\FunctionTok{length}\NormalTok{(}\FunctionTok{unique}\NormalTok{(df.GRUPO)) }\OperatorTok{{-}} \FloatTok{1}\NormalTok{, df\_residual],}
\NormalTok{    SS }\OperatorTok{=}\NormalTok{ [}\FunctionTok{sum}\NormalTok{(terms.cols[}\FloatTok{3}\NormalTok{]}\OperatorTok{.\^{}}\FloatTok{2}\NormalTok{), ss\_residual],}
\NormalTok{    MS }\OperatorTok{=}\NormalTok{ [}\FunctionTok{sum}\NormalTok{(terms.cols[}\FloatTok{3}\NormalTok{]}\OperatorTok{.\^{}}\FloatTok{2}\NormalTok{)}\OperatorTok{/}\NormalTok{(}\FunctionTok{length}\NormalTok{(}\FunctionTok{unique}\NormalTok{(df.GRUPO)) }\OperatorTok{{-}} \FloatTok{1}\NormalTok{), ss\_residual}\OperatorTok{/}\NormalTok{df\_residual],}
\NormalTok{    F }\OperatorTok{=}\NormalTok{ [terms.cols[}\FloatTok{4}\NormalTok{][}\FloatTok{2}\NormalTok{], }\ConstantTok{missing}\NormalTok{],}
\NormalTok{    p }\OperatorTok{=}\NormalTok{ [terms.cols[}\FloatTok{5}\NormalTok{][}\FloatTok{2}\NormalTok{], }\ConstantTok{missing}\NormalTok{]}
\NormalTok{)}

\FunctionTok{println}\NormalTok{(df\_anova)}
\NormalTok{CSV.}\FunctionTok{write}\NormalTok{(}\StringTok{"resultados\_anova\_Julia.csv"}\NormalTok{, df\_anova)}
\end{Highlighting}
\end{Shaded}

En comparación con R y Python, Julia se beneficia de la
\textbf{compilación JIT}, logrando tiempos de ejecución muy
competitivos, especialmente con grandes volúmenes de datos. El código es
igualmente compacto (2--4 líneas por modelo) y la precisión de los
resultados es consistente con R dentro de la tolerancia de doble
precisión.

\begin{center}\rule{0.5\linewidth}{0.5pt}\end{center}

\subsection{\texorpdfstring{📌 \textbf{Comentario
final}}{📌 Comentario final}}\label{comentario-final}

Los bloques de código muestran cómo cada lenguaje implementa
\textbf{regresión} y \textbf{ANOVA} de forma declarativa, legible y
automatizada. Aunque R destaca por su sintaxis minimalista para
estadística, Python sobresale por su ecosistema multipropósito y Julia
por su rendimiento optimizado. Esta base permitió medir de forma
comparable \textbf{tiempo de ejecución}, \textbf{uso de memoria},
\textbf{precisión numérica} y \textbf{longitud de código}, generando
datos consistentes para el \textbf{ANOVA factorial de dos factores} de
esta investigación.

Dependientes: Tiempo(s) y Memoria(MB) → multivariado

Factor: Lenguaje → categórico de 3 niveles

N: 30 réplicas × 3 lenguajes = 90 observaciones independientes

Linealidad / Normalidad / Homogeneidad: debes verificarlo gráficamente y
con los tests habituales (Shapiro--Wilk, Box's\,M), pero la estructura
de tu tabla y el tamaño muestral son adecuados.




\end{document}
